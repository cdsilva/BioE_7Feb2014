\section{Pre-Aligned Images: Principal Component Analysis}

\begin{frame}{Principal Component Analysis (PCA)}
  \begin{itemize}
  \item The goal of PCA \footcite{shlens2005tutorial} is to find the direction(s) of maximum variance in a data set
  \item In our case, we are assuming that there is one direction with significantly more variance than the others
  \item We assume that if we project our data along this line, we will get an embedding that is correlated with time
  \end{itemize}
\end{frame}

\begin{frame}{PCA: Covariance Matrix}
  \begin{itemize}
  \item We have $n$ {\bf data points}, each in $d$ dimensions, denoted $x_1, x_2, \dots, x_n \in \mathbb{R}^d$, where $x_{i,j}$ denotes the $j^{th}$ component of data point $i$
  \item We first define the {\bf mean-centered} data points $\hat{x}_i$ as $$\hat{x}_{i,j} = x_{i,j} - \frac{1}{n} \sum_{i=1}^{n} x_{i,j}$$
  \item Let $X \in \mathbb{R}^{n \times d}$ denote the data matrix, where the $i^{th}$ row of $X$ contains $\hat{x}_i$
  \item The {\bf covariance matrix} $R \in \mathbb{R}^{d \times d}$ can be estimated as $$R = \frac{1}{n-1} X^T X$$
  \end{itemize}
\end{frame}

\begin{frame}{PCA: Eigendecomposition}
\begin{columns}
\begin{column}{0.7\textwidth}
  \begin{itemize}
    \item We then compute the {\bf eigenvectors} $v_1, v_2, \dots, v_d$ and {\bf eigenvalues} $\lambda_1, \lambda_2, \dots, \lambda_d$ of $R$ (all eigenvectors are normalized to 1)
    \item By construction, $R$ is symmetric and positive semi-definite and is therefore guaranteed to have real non-negative eigenvalues and orthogonal real eigenvectors
    \item We {\bf order} the eigenvector/eigenvalue pairs such that $\lambda_1 \ge \lambda_2 \ge \dots \ge \lambda_d$
    \item $v_1, v_2, \dots, v_d$ are called the {\bf principal components}, and $\lambda_j$ measures the {\bf variance} captured by principal component $j$
  \end{itemize}
  \end{column}

  \begin{column}{0.3\textwidth}
        \includegraphics[width=\textwidth]{../Stas_data/group_meeting_5Feb2013/PCA_spectrum1.jpg}\\
        {\tiny The PCA eigenvalue spectrum; each eigenvalue measures the energy captured by the corresponding eigenvector \par}

        \vspace{0.3 in}

        \includegraphics[width=\textwidth]{../Stas_data/group_meeting_5Feb2013/PCA_mode1.jpg}\\
        {\tiny The first principal component \par}

  \end{column}
  \end{columns}
\end{frame}

\begin{frame}{PCA: Ordering and Projections}
    \begin{itemize}
        \item We can calculate the projection of data point $i$ onto principal component $j$ as $a_{i,j} = \langle \hat{x}_i, v_j \rangle = \sum_{k=1}^{d} \hat{x}_{i,k} v_{j,k}$
        \item $a_{i,j}$ can be viewed as a measure of how much principal component $j$ is represented in data point $i$
        \item We can order the data by $a_{i,1}$
        \item Each principal component defines an important ``direction'' in our data; we can compare data sets by comparing their principal components
    \end{itemize}

    \centering
    \includegraphics[height=0.3\textheight]{../Stas_data/group_meeting_5Feb2013/data1_scrambled.jpg}
    \includegraphics[height=0.3\textheight]{../Stas_data/group_meeting_5Feb2013/coeff_scrambled.jpg}\\
    {\Tiny The scrambled data (left) and the projection coefficient for each data point onto the first principal component (right).\\
    We will sort the data by the values of the projection coefficients. \par}

\end{frame}

\begin{frame}{Results}
  \centering
    Below are three replicate data sets.\\
    You can see the scrambled data (left), unscrambled data using PCA (middle), and first PCA mode (right)

    \includegraphics[width=0.3\textheight]{../Stas_data/group_meeting_5Feb2013/data1_scrambled.jpg}
    \includegraphics[width=0.3\textheight]{../Stas_data/group_meeting_5Feb2013/data1_unscrambled.jpg}
    \includegraphics[width=0.3\textheight]{../Stas_data/group_meeting_5Feb2013/PCA_mode1.jpg}\\
    \includegraphics[width=0.3\textheight]{../Stas_data/group_meeting_5Feb2013/data2_scrambled.jpg}
    \includegraphics[width=0.3\textheight]{../Stas_data/group_meeting_5Feb2013/data2_unscrambled.jpg}
    \includegraphics[width=0.3\textheight]{../Stas_data/group_meeting_5Feb2013/PCA_mode2.jpg}\\
    \includegraphics[width=0.3\textheight]{../Stas_data/group_meeting_5Feb2013/data3_scrambled.jpg}
    \includegraphics[width=0.3\textheight]{../Stas_data/group_meeting_5Feb2013/data3_unscrambled.jpg}
    \includegraphics[width=0.3\textheight]{../Stas_data/group_meeting_5Feb2013/PCA_mode3.jpg}

\end{frame}