\section[Factoring Out Symmetries in Two--Dimensional~Images]{Factoring Out Symmetries in Two--Dimensional~Images}

\begin{frame}{Images Instead of Concentration Profiles}
	
	\centering
    As a preprocessing step, we have to take fluorescent images of the embryo~cross-sections and convert them to concentration profiles on a ring.
    
    \centering
    \begin{tikzpicture}
        \node (dpERKimage) {\includegraphics[width=0.25\textwidth]{drosophila_dpERK}};
        		\draw[gray,<->] (dpERKimage.south west) --  (dpERKimage.south east) node[below,midway] { \tiny $100 \mu m$};

        \node[right=of dpERKimage] (dpERKprofile) {\includegraphics[width=0.25\textwidth]{circle_profile}};
        \draw[->] (dpERKimage)--(dpERKprofile);
    \end{tikzpicture}
    
    \begin{itemize}
    	\item DMAPS only requires a distance metric.
    	\item Angular synchronization only requires us to compute pairwise alignments.
    	\item Therefore, we can do the same analysis {\em directly} on~the~raw~two--dimensional~fluorescent~images.
    	\item We will factor out {\bf translations and rotations}.
    	\item We will use the Euclidean distance between the intensities of the image pixels as our distance metric.
    \end{itemize}
    
	\end{frame}

\begin{frame}{Ordering Fluorescent Images Using Vector Diffusion Maps}

	\centering
    We align and order the images using vector diffusion maps.
    \vspace{0.1in}

	\drawunordered
    
    \drawdownarrow
    	
    \begin{minipage}{0.8\textwidth}
    \centering
	{\scriptsize {\bf VDM} \\Pairwise alignments account for translations and rotations of images \\ Euclidean distance between image pixels as metric \par}
    \end{minipage}
    
    \drawdownarrow
    
    \only<1>{\movievdm{0.2\textwidth}}
		
	\only<2>{\draworderedvdm
	
	\vspace{0in}}
    
    
\end{frame}

\begin{frame}{Scattering Transform as Image Features}

{\small 
\begin{itemize}
	\item Thus far, we have focused on {\em aligning} the images so that we can factor out the~relevant symmetries before doing further analysis. 
	%
	\item Another option is to use {\em features} of the images that are invariant to translations and rotations, and use distances between these features in our DMAPS calculations.
	%
	\item 
	We will use the scattering transform \footcite{bruna2012invariant} as features of our images.
\end{itemize}
\par}
\vspace{-0.15in}
\begin{minipage}{0.55\textwidth}
\begin{block}{{\small Scattering transform is like the Fourier transform \par}}
\centering 
\includegraphics[width=0.75\textwidth]{scat_images/shapeimage_2}\\
{\tiny (Left) $f$ = indicator of a square, (Center) Modulus of the Fourier transform of $f$, (Right) Scattering transform of $f$ \par}
\end{block}

\vspace{-0.1in}
\begin{block}{{\small Scattering transform is robust to deformations \par}}
\centering
\animategraphics[width=0.8\textwidth,autoplay,loop]{5}{scat_images/animate_}{0}{3} 

{\tiny (Left) $f$ = Gabor atom of varying frequency and direction, (Center) Modulus of the Fourier transform of $f$, (Right) Scattering transform of $f$ \par}
\end{block}

\end{minipage}
\hfill
\begin{minipage}{0.4\textwidth}
\begin{block}{{\small Scattering transform can discriminate higher-order structures \par}}
\includegraphics[height=0.15\textheight]{scat_images/texture8-1}
\includegraphics[height=0.15\textheight]{scat_images/texture8-f1}
\includegraphics[height=0.15\textheight]{scat_images/shapeimage_3}

\includegraphics[height=0.15\textheight]{scat_images/texture8-eq}
\includegraphics[height=0.15\textheight]{scat_images/texture8-f1eq}
\includegraphics[height=0.15\textheight]{scat_images/shapeimage_4}\\
{\tiny (Left) $X_1, X_2$ = Two realizations of stationary textures. $X_2$ is obtained by equalizing random white noise according to the spectrum of $X_1$. (Center) Power spectrum of $X_1$, $X_2$. (Right) Scattering transform of $X_1$, $X_2$. High order scattering coefficients discriminate between the Gaussian process $X_2$ and the non-Gaussian $X_1$. \par}
\end{block}

\end{minipage}

\end{frame}

\begin{frame}{Ordering of Two-Dimensional Images}

	{\small 
	\begin{itemize}
		\item We compute the scattering transform coefficients for each image in our data set.
		\item We use the Euclidean distance {\em between the scattering transform coefficents} in our DMAPS calculation.
	\end{itemize}
    \par}

	\vspace{0.1in}
	
	\drawunordered
    
    	\drawdownarrow 
    	
    \begin{minipage}{0.5\textwidth}
    \centering
	{\scriptsize Scattering transform + DMAPS \par}
    \end{minipage}\\
    \drawdownarrow
    
	\draworderedscat 
    
\end{frame}

    
    

    